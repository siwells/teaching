\documentclass[letterpaper,11pt]{../resources/texMemo}

\usepackage{graphicx}
\usepackage{url}

\student{}
\matric{}
\grade{  \%} 
\marker{Simon Wells}
\subject{Feedback on SET08101 Coursework Assignment \#2}

\logo{\includegraphics[width=0.25\textwidth]{../resources/enu_logo.png}}

\begin{document}
\maketitle

\paragraph{Written Feedback:}





\paragraph{Information about your grade:} The marking scheme is devised so as to reward those who go beyond the core taught material by integrating their own self-directed learning and discoveries. A reasonable attempt at a difficult application is likely to attract more marks than a complete implementation of a simple application. As a general rule, the more functionality, the better the mark, however your functionality should be consistent with a cohesive overall design.
\begin{description}
\item

%[70-100\%] A submission in this mark band will demonstrate that you have gone beyond the core learning for the module and have actively pursued your own learning path. Your submission will include a blog framework that goes beyond the core techniques discussed in class and lab sessions and that offers an excellent level of functionality with a rewarding user experience. You will have evaluated your design using appropriate techniques. You will have implemented more advanced features that have not been specifically covered in the practical sessions and which you will have investigated for yourself. Your design and code will be excellent. All HTML, CSS, and JavaScript will be well organised. Your report will be comprehensive, very well written and presented, and will correctly reference all the material you have used. This is likely to include textbooks, online forums and tutorials and some of the suggested reading for the module.

%[60-69\%] To achieve a mark in this band you will have developed a site with very good functionality, offering the user the ability to add, edit, remove, and view blog posts. Your server component will persist your posts using an appropriate strategy. You will have protected, using an appropriate methodology, any features, such as adding, editing or deleting a post, that could be detrimental to your users experience if misused. Your site will have a pleasing design, making very good use of appropriately selected HTML, CSS, Javascript features in order to provide a pleasing user experience. Your report will address all the necessary sections effectively, be very well written, clearly presented, and will reference all materials you have used.

%[50-59\%] A submission graded into this mark band will indicate that you have developed a blog framework that is less ambitious in its functionality. Your user will be able to add, edit and view blog posts. Posts will be persisted using an appropriate strategy. Your site will have solid design and provide an acceptable user experience. Your report will be well written and will reference the material you have used.

%[40-49\%] To achieve a mark in this band you must have developed a working Node.JS server component that serves up an appropriate HTML interface to your blog framework. Your user must be able to navigate between your pages, to read the latest blog post and add new ones. Your design will be rudimentary but a basic usability requirement is that other users (aside from yourself) must be able to navigate your web-site. A submission in the grade band may be based on an extension of the practical work covered in class. Your report will adequately describe your work.

\end{description}

\end{document}
